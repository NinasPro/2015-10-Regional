\documentclass{article}
\usepackage[utf8]{inputenc}
\usepackage{fullpage}

\title{Colección de Láminas}

\begin{document}
\maketitle

Jorge es un niño que tiene una única pasión en la vida, coleccionar láminas para
completar sus álbumes favoritos. Cada álbum está compuesto por cierta cantidad
de láminas y para completarlo es necesario conseguirlas todas. El último álbum
que Jorge logró completar fue el de los Lokómones. En la imagen de abajo se
muestran algunas de las láminas del álbum.

Para completar los álbumes Jorge debe comprar sobres con láminas. Cada uno de
los sobres contiene 5 láminas, pero Jorge no sabe que láminas le saldrán antes
de comprarlos. Es por esto que siempre debe comprar muchos sobres extras, pues
es muy probable que al comprarlos le salgan láminas que ya tenía. Esto es un
problema pues los sobres son bastante caros y las láminas repetidas no le sirven
para nada, ni siquiera para intercambiarlas con otras personas, pues Jorge no
tiene amigos.

Luego de completar su último álbum, el de los Lokómones, Jorge se enteró que
también era posible comprar láminas sueltas escogiendo las que quisiera. Si
Jorge se hubiera enterado antes de esto no habría acumulado tantas láminas
repetidas, pues simplemente podría haber comprado las láminas que le faltaban.

Jorge además se pregunta si comprando láminas sueltas podría haber gastado menos
dinero. No es tan sencillo darse cuenta de esto, pues algunas láminas, al ser más
``raras'', son más caras que otras al comprarlas sueltas. Jorge cree que habría
sido conveniente comprar sobres hasta cierto punto y luego de eso comprar
sueltas todas las láminas que no alcanzó a recolectar con los sobres. Además de
coleccionar láminas Jorge no tiene muchas otras habilidades y le está constando
encontrar cuál hubiera sido el momento óptimo donde dejar de comprar sobres. Por
suerte Jorge es un niño muy ordenado y anotó las láminas que le salieron en
cada uno de los sobres que compró. ?`Podrías ayudarlo?

El álbum consiste en $N$ láminas y por simplicidad nos referiremos a ellas con
números del 1 al $N$. Cada sobre contiene 5 láminas y tiene un precio $P$.
Además sabes el precio asociado a la compra de cada lámina suelta y las láminas
que salieron en cada sobre que compró Jorge. Tu tarea es encontrar la mínima
cantidad de dinero que Jorge podría haber gastado para completar todo el álbum
si hubiera dejado de comprar sobres en algún punto y luego hubiera comprado
láminas sueltas.

\section*{Input}

La primera línea de la entrada contiene tres enteros positivos separados por un
espacio. Estos corresponden respectivamente a la cantidad de láminas del álbum
($N$), la cantidad de sobres ($S$) y el precio de cada sobre ($P$). La siguiente
línea contiene $N$ enteros positivos separados por espacios que corresponden
al precio de cada lámina. El primer entero al precio de la lámina 1, el segundo
al precio de la lámina 2, etc. Las siguientes $S$ líneas contienen la
descripción de cada sobre que compró Jorge. Cada una de estas líneas contiene 5
enteros entre 1 y $N$ describiendo las láminas que salieron en el sobre.

\section*{Salida}

Debes imprimir un sólo entero correspondiente a la mínima cantidad de dinero que
podría haber gastado Jorge si solo hubiera comprado sobres hasta cierto punto y
luego hubiera comprado el resto de las láminas sueltas.

\section*{Subtareas}

\begin{itemize}
  \item $S=1$: responder la suma de cada lámina por separado excepto las que salieorn en el sobre.
  \item $P=1$ y en los sobres nunca sale una lámina repetida: responder $S$
  \item Caso general recorrer cada sobre y mantener en una variable con el precio que saldría comprar el resto de las láminas mas lo que han costado los sobres, quedarse con la menor suma.
\end{itemize}

 

\end{document}
