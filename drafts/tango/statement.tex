\documentclass{article}
\usepackage[utf8]{inputenc}
\usepackage{fullpage}

\title{Tango}

\begin{document}
\maketitle

Después de mucho pensarlo, Nelman ha decidido tomar clases de tango. Después de
cotizar en clubes, centros culturales y gimnasios, ha resuelto aprender en
Overflow Club International, tanto por su ubicación accesible como por sus
excelentes precios.

Al llegar a la primera lección y tras los ejercicios de calentamiento, Vardieri,
el instructor, indicó a los asistentes que formen parejas. Aquí es donde 
empezaron las dificultades, pues Vardieri tiene ciertas manías estéticas. Para
Vardieri, los hombres siempre deben bailar junto a una mujer, y además, ésta no
debe ser de mayor estatura que el hombre (una manía bastante machista, pensó
Nelman).

Vardieri no permitirá que la clase continúe si existe alguna pareja que no
cumpla con su criterio. Así que cualquier pareja que no cumpla con el criterio
deberá retirarse de la sala. A pesar de tener esta extraña manía, a Vardieri le
interesa que la gente aprenda a bailar y le gustaría tener la mayor cantidad
de parejas posibles. Como hay mucha gente le está costando definir cuales deben
ser las parejas. ¿Puedes ayudar a Vardieri a definir las parejas?
No importa que Nelman quede sin pareja.

\section*{Puntaje}
Este problema no contiene subtareas. Tu solución será evaluada dependiendo de
cuantas parejas puedes formar. Para un caso de prueba, si $M$ es el máximo de
parejas que es posible formar y $P$ es la cantidad de parejas que pudiste formar
recibirás $(1-\frac{M-P}{M}\times 100$) puntos. El puntaje total será el
promedio de todos los casos de prueba. Si tu solución no corresponde a un
emparejamiento válido recibirás 0 puntos.

\section*{Input}
%FIXME no hay sangría
La primera línea de la entrada contiene dos enteros $H$ y $W$ que corresponden a
la cantidad de hombres y mujeres respectivamente. A continuación, siguen dos líneas.
La primera contiene $H$ enteros mayores que cero $x_1 \dots x_H$ que corresponden
a las alturas de cada hombre. La segunda linea contiene $W$ enteros $y_1 \dots y_W$, 
que corresponden a las alturas de las mujeres.

\section*{Salida}
%FIXME no hay sangría
Debes imprimir cada pareja en una línea. Cada línea es de la forma $i$ $j$, que
indica que el $i$-ésimo hombre (dentro de la lista de alturas) bailará con la
$j$-ésima mujer (de la lista de alturas).


\end{document}
