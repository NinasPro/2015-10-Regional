\documentclass{article}
\usepackage[utf8]{inputenc}
\usepackage{fullpage}

\title{Tango}

\begin{document}
\maketitle

Después de mucho pensarlo, Nelman ha decidido tomar clases de tango. Después de
cotizar en clubes, centros culturales y gimnasios, ha resuelto aprender en
Overflow Club International, tanto por su ubicación accesible como por sus
excelentes precios.

Al llegar a la primera lección y tras los ejercicios de calentamiento, Vardier,
el instructor, indicó a los asistentes que formen parejas. Aquí es donde 
empezaron las dificultades, pues Vardier tiene ciertas manías estéticas. Para
Vardier, los hombres siempre deben bailar junto a una mujer, y además, ésta no
debe ser de mayor estatura que el hombre (una manía bastante machista, pensó
Nelman).

Por fortuna, el número de hombres y de mujeres es el mismo, pero aun así,
Vardier no permitirá que la clase continúe si al menos la mitad de las parejas
no se forma acorde a su particular criterio, espantado por tal aberración de
forma. ¿Puedes ayudar a Nelman a definir las parejas de manera tal que Vardier
esté contento y así continúe su clase?

\section*{Input}
%FIXME no hay sangría
La primera línea de la entrada contiene un único entero $N$, que corresponde al
número de parejas en la clase. A continuación, siguen dos líneas. Éstas
contienen cada una $N$ números $h_1 \dots h_N$ y $m_1 \dots m_N$, separados por
espacios. Éstos corresponden a las alturas de los hombres ($h_i$) y mujeres
($m_i$) participantes de la clase.

\section*{Salida}
%FIXME no hay sangría
Debes imprimir cada pareja en una línea. Cada línea es de la forma $i$ $j$, que
indica que el $i$-ésimo hombre (dentro de la lista de alturas) bailará con la
$j$-ésima mujer (de la lista de alturas).

Si no es posible formar parejas tal que al menos
$\left\lceil\frac{N}{2}\right\rceil$ de ellas cumplan la exigencia de Vardieri,
se debe imprimir una sola línea con la palabra ``\verb+aberracion+'' en lugar
de la lista de parejas.


\end{document}
