\documentclass{article}
\usepackage[utf8]{inputenc}
\usepackage{fullpage}

\title{Marraquetas}

\begin{document}
\maketitle

Un padre manda a su hijo a comprar una cierta cantidad de marraquetas. El irá a comprar en su panadería favorita en el ACM (Área de Consumo/Compra de Marraquetas), que tiene mesas para que la gente tome desayuno y bla bla bla o lo que sea. El niño no sabe cuánto es una marraqueta pues siempre había comprado hallullas, así que antes de ir a comprar el niño pregunta a su padre cuánto pan es una marraqueta y le dice que según el recuerda en la panaderías las venden de a 4 panes pegados y que cada uno de ellos es una marraqueta. El niño cree que en realidad esos 4 panes pegados deben ser una sola marraqueta. Sin embargo, cuando llega a la panadería, ve que alguien compra 2 marraquetas y le dan el conjunto de 4 panes pegados. Dada la cantidad de marraquetas que quiere el papá según su creencia, dar como output la cantidad de marraquetas que el niño debe pedirle al panadero, que solo puede vender una cantidad entera de marraquetas.
En caso de no poderse comprar la cantidad exacta, redondear hacia arriba.

\section*{Entrada}
Input: n (cantidad de marraquetas según el padre)

\section*{Salida}
Output: x (cantidad que debe pedir en la panadería).


\section*{Subtareas}
Subtareas:
No es necesario redondear. $0 < n < 100000$. 60 puntos.
Podría ser necesario redondear. $0 < n < 100000$. 25 puntos.


\section*{Solución Esperada}
Solución esperada: Dividir por 2 redondeando hacia arriba.


 

\end{document}
