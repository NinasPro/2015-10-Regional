\documentclass{article}
\usepackage[utf8]{inputenc}
\usepackage{fullpage}

\title{Marraquetas}

\begin{document}
\maketitle

En la casa de Alejandra tendrán un asado familiar así que su padre la ha mandado
a comprar cierta cantidad de marraquetas para poder hacer choripanes. Alejandra
no sabe exactamente cuanto pan es `una marraqueta' pues siempre ha comprado
hallullas, así que antes de partir pregunta a su padre cuanto pan es `una
marraqueta'. Su padre le responde que según recuerda, en las panaderías se
encuentran como 4 rollos de pan pegados y que cada uno de estos rollos es
`una marraqueta'. Alejandra queda extrañada, pues si los cuatro rollos vienen
pegados deberían en su conjunto ser llamados `una marraqueta'. Para no discutir
Alejandra simplemente acepta la definición de marraqueta de su padre y se dirige
a la panadería para cumplir con la orden.

Por si no fuera ya el colmo al llegar a la panadería Alejandra queda aún más
confundida, pues ve a un cliente comprar 2 marraquetas y salir con el conjunto
de 4 rollos pegados. Al parecer en la panadería siguen la definición de la
Asociación de Consumo de Marraquetas (ACM) y consideran que `una marraqueta' son
dos rollos de pan pegados. 

En la panadería solo pueden vender una cantidad entera de marraquetas y
Alejandra debe pedirlas según la definición de la ACM\. Dada la cantidad de
marraquetas que quiere el padre según su definición, `?podrías ayudar a
Alejandra a saber cuantas marraquetas pedirle al panadero?. Notar que no siempre
es posible comprar la cantidad exacta de marraquetas según la definición del
padre.


% Un padre manda a su hijo a comprar una cierta cantidad de marraquetas. El irá a comprar en su panadería favorita en el ACM (Área de Consumo/Compra de Marraquetas), que tiene mesas para que la gente tome desayuno y bla bla bla o lo que sea. El niño no sabe cuánto es una marraqueta pues siempre había comprado hallullas, así que antes de ir a comprar el niño pregunta a su padre cuánto pan es una marraqueta y le dice que según el recuerda en la panaderías las venden de a 4 panes pegados y que cada uno de ellos es una marraqueta. El niño cree que en realidad esos 4 panes pegados deben ser una sola marraqueta. Sin embargo, cuando llega a la panadería, ve que alguien compra 2 marraquetas y le dan el conjunto de 4 panes pegados. Dada la cantidad de marraquetas que quiere el papá según su creencia, dar como output la cantidad de marraquetas que el niño debe pedirle al panadero, que solo puede vender una cantidad entera de marraquetas.
% En caso de no poderse comprar la cantidad exacta, redondear hacia arriba.

\section*{Entrada}
La entrada consiste en una línea con un único entero $N$ que representa la
cantidad de marraquetas que Alejandra debe comprar según la definición de su
padre.

\section*{Salida}
Debes imprimir una línea con un único entero correspondiente a la menor cantidad
de marraquetas según la definición de la ACM que Alejandra debe comprar para
llevar a su padre la cantidad de marraquetas que solicitó según su definición.


\section*{Subtareas}
\begin{itemize}
\item Siempre es posible comprar la cantidad exacta de marraquetas. $0 \leq N \leq 100$. 50 puntos.
\item $0 \leq N \leq 100$ y no hay restricciones adicionales. 50 puntos.
\end{itemize}


\section*{Solución Esperada}
Solución esperada: Dividir por 2 redondeando hacia arriba.


 

\end{document}
