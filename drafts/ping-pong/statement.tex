\documentclass{article}
\usepackage[utf8]{inputenc}
\usepackage{fullpage}

\title{Partido de Ping pong}

\begin{document}
\maketitle

Jota Pe y Lehmann quieren jugar una partida de tenis de mesa, sin embargo, como no es una partida profesional ellos deciden crear sus propias reglas
la primera regla es que gana el primer jugador en llegar a N puntos, y la segunda regla es que cada jugador parte y conserva su saque hasta perder
un punto, en donde empieza a sacar el otro jugador, esta regla genera que hayan rachas durante el partido, estas rachas son constantes

\section*{Entrada}
Entrada

una linea con 3 numeros A, B, N, y una letra P, el numero A representa el largo de las rachas de Jota Pe, y la letra B el largo de las rachas de Lehmann
y N es la cantidad de puntos a las que debe llegar alguno de los jugadores para ganar, por ultimo la letra P representa quien parte, si la letra es una J
entonces parte JotaPe si es una L parte Lehmann


\section*{Salida}

Imprima en una linea ``Jota Pe'' sin comillas o ``Lehmann'' dependiendo de quien sea el ganador

\section*{Casos de prueba}

Input:

3 5 6 J

Output:

Jota Pe

-------------

Input:

3 5 6 L

Output:

Lehmann

Idea: Una opción es simular $O(N)$ por lo que puede ser una de las subtareas, pero la idea principal es calcular cuantas jugadas necesita cada jugador
para ganar la partida, si parte A, la idea es que gana A si es que las rachas que necesita para ganar son <= que la del que no parte



\end{document}
