\documentclass{oci}
\usepackage[utf8]{inputenc}
\usepackage{lipsum}

\title{Parejas para el Tango}

\begin{document}
\begin{problemDescription}
Luego de meditarlo por mucho tiempo, Nelman ha decidido tomar clases de tango.
Después de cotizar en clubes, centros culturales y gimnasios, ha determinado inscribirse en el famoso Okura Club International, tanto por su ubicación como por sus excelentes precios.

Al llegar a la primera lección y tras los ejercicios de calentamiento, Vardieri, el instructor, indicó a los asistentes que debían formar parejas.
Aquí es donde empezaron las dificultades, pues Vardieri tiene ciertas manías estéticas muy extrañas.
Para Vardieri, los hombres siempre deben bailar junto a una mujer, y además, esta no puede ser de mayor estatura que el hombre (una manía bastante retrograda, pensó Nelman).

Vardieri no permitirá que la clase continúe mientras no se formen las parejas y no tolerará bajo ningún motivo a una pareja que no cumpla con su criterio.
Mientras todos se miran sin saber que hacer Vardieri espera tranquilo en una esquina de la sala.
Tratando de salvar la clase, Nelman propone formar la mayor cantidad de parejas posibles aunque esto signifique que algunos se queden sin poder bailar.
Todos están de acuerdo con esto pero aún siguen sin saber cómo formar las parejas de forma que se arme la mayor cantidad posible.

En la sala hay la misma cantidad de mujeres que de hombres y además no hay dos hombres con la misma estatura ni dos mujeres con la misma estatura.
Esto es, todos los hombres son de estaturas distintas y lo mismo para las mujeres.
Tu tarea es ayudar a los asistentes para que formen la mayor cantidad de parejas posibles.
Notar que puede haber más de una forma de asignar las parejas para que esto se cumpla y si este es el caso cualquier forma servirá.

\end{problemDescription}

\begin{inputDescription}
La primera línea de la entrada contiene un entero positivo $N$ que corresponde a la cantidad de hombres y de mujeres.
La segunda contiene $N$ enteros positivos que describen las alturas de los hombres.
Similarmente la tercera línea contiene $N$ enteros positivos que describen las alturas de las mujeres.
Debes asumir que tanto en la lista de hombres como de mujeres no existirán números repetidos.
\end{inputDescription}

\begin{outputDescription}
Debes imprimir cada pareja en una línea separada.
Cada línea debe contener dos enteros $H$ y $M$, correspondientes respectivamente a las alturas del hombre y de la mujer en la pareja.
\end{outputDescription}

\begin{scoreDescription}
  \score{30} Se probarán varios casos donde el número de hombres y mujeres es menor o igual que 2 ($0 < N \leq 2$)
  \score{30} Se probarán varios casos donde para cada mujer existe un hombre de la misma estatura.
  \score{40} Se probarán varios casos donde no hay restricciones.
\end{scoreDescription}

\begin{sampleDescription}
% \sampleIO{sample}
% \sampleIO{sample}
\end{sampleDescription}

\end{document}
