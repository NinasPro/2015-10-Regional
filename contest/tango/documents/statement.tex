\documentclass{oci}
\usepackage[utf8]{inputenc}
\usepackage{lipsum}

\title{Parejas para el Tango}

\begin{document}
\begin{problemDescription}
Luego de meditarlo por mucho tiempo, Nelman ha decidido tomar clases de tango.
Después de cotizar en clubes, centros culturales y gimnasios, ha determinado
inscribirse en el famoso Okura Club International, tanto por su ubicación como
por sus excelentes precios.

Al llegar a la primera lección y tras los ejercicios de calentamiento,
Vardieri, el instructor, indicó a los asistentes que debían formar parejas.
Aquí es donde empezaron las dificultades, pues Vardieri tiene ciertas manías
estéticas muy extrañas. Para Vardieri, los hombres siempre deben bailar junto a
una mujer, y además, ésta no puede ser de mayor estatura que el hombre (una
manía bastante retrógrada, pensó Nelman).

Vardieri no permitirá que la clase continúe mientras no se formen las parejas y
no tolerará bajo ningún motivo a una pareja que no cumpla con su criterio.
Mientras todos se miran sin saber que hacer, Vardieri espera tranquilo en una
esquina de la sala. Tratando de salvar la clase, Nelman propone formar la mayor
cantidad de parejas posibles aunque esto signifique que algunos se queden sin
poder bailar. Todos están de acuerdo con esto, pero aún siguen sin saber cómo
formar las parejas de forma que se obtenga la mayor cantidad posible.

En la sala hay igual número de mujeres que de hombres, y además, no hay dos
hombres de la misma estatura ni dos mujeres de la misma estatura. Es decir,
todos los hombres son de estaturas distintas, y lo mismo para las mujeres. Tu
tarea es ayudar a los asistentes para que formen la mayor cantidad de parejas
posibles. Notar que puede haber más de una forma de asignar las parejas para
que esto se cumpla, y si este es el caso, cualquier forma servirá.

\end{problemDescription}

\begin{inputDescription}
La primera línea de la entrada contiene un entero positivo $N \geq 1$ que
corresponde a la cantidad de hombres y de mujeres. La segunda contiene $N$
enteros positivos que describen las alturas $h_i$ de los hombres ($50 \leq h_i
\leq 210$) en centímetros. Similarmente la tercera línea contiene $N$ enteros
positivos que describen las alturas $m_i$ de las mujeres ($50 \le m_i \le 210$)
en centímetros. Debes asumir que tanto en la lista de hombres como de mujeres
no habrá números repetidos.
\end{inputDescription}

\begin{outputDescription}
Debes imprimir cada pareja en una línea diferente.
Cada línea debe contener dos enteros $H_i$ y $M_i$ separados por un espacio,
los que corresponden respectivamente a las alturas del hombre y de la mujer de
una pareja.
\end{outputDescription}

\begin{scoreDescription}
  \score{30} Se probarán varios casos donde el número de hombres y mujeres es menor o igual que 2 ($1 \leq N \leq 2$).
  \score{30} Se probarán varios casos donde para cada mujer existe un hombre de igual estatura.
  \score{40} Se probarán varios casos donde no hay restricciones aparte de las de enunciado.

\textbf{Nota}: La restricción de la subtarea 2 no aplica para la subtarea 1.
\end{scoreDescription}

\begin{sampleDescription}
 \sampleIO{sample1}
 \sampleIO{sample2}
\end{sampleDescription}

\end{document}
