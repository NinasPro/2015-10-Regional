\documentclass{oci}
\usepackage[utf8]{inputenc}
\usepackage{lipsum}

\title{?`Cuánto pan es una Marraqueta?}

\begin{document}
\begin{problemDescription}
En la casa de Alejandra tendrán un asado familiar así que su padre la ha enviado
a comprar marraquetas para poder hacer choripanes. Alejandra
no sabe exactamente cuanto pan es \textit{una marraqueta} pues siempre ha comprado
hallullas, así que antes de partir pregunta a su padre cuanto pan es \textit{una
marraqueta}. Su padre le responde que según recuerda, en las panaderías se
encuentran como 4 rollos de pan pegados y que cada uno de estos rollos es
\textit{una marraqueta}. Alejandra queda extrañada, pues si los cuatro rollos vienen
pegados deberían en su conjunto ser llamados \textit{una marraqueta}. Para no discutir
Alejandra simplemente acepta la definición de marraqueta de su padre y se dirige
a la panadería para cumplir con la orden.

Por si no fuera ya el colmo al llegar a la panadería Alejandra queda aún más
confundida, pues ve a un cliente comprar 2 marraquetas y salir con el conjunto
de 4 rollos pegados. Al parecer en la panadería siguen la definición de la
Asociación de Consumo de Marraquetas (ACM) y consideran que \textit{una marraqueta} son
dos rollos de pan pegados. 

En la panadería solo pueden vender una cantidad entera de marraquetas y
Alejandra debe pedirlas según la definición de la ACM\@. Dada la cantidad de
marraquetas que quiere el padre según su definición, ?`podrías ayudar a
Alejandra a saber cuantas marraquetas pedirle al panadero? Notar que no siempre
es posible comprar la cantidad exacta de marraquetas según la definición del
padre.
\end{problemDescription}

\begin{inputDescription}
La entrada consiste en una línea con un único entero $N$ que representa la
cantidad de marraquetas que Alejandra debe comprar según la definición de su
padre.
\end{inputDescription}

\begin{outputDescription}
Debes imprimir una línea con un único entero correspondiente a la menor cantidad
de marraquetas según la definición de la ACM que Alejandra debe comprar para
llevar a su padre la cantidad de marraquetas que solicitó según su definición.
\end{outputDescription}

\begin{scoreDescription}
  \score{50} Se probarán varios casos donde $0<N\leq 100$ y siempre es posible comprar la cantidad exacta de marraquetas. 
  \score{50} Se probarán varios casos donde $0<N\leq 100$ y sin restricciones adicionales.
\end{scoreDescription}

\begin{sampleDescription}
% \sampleIO{sample}
% \sampleIO{sample}
\end{sampleDescription}

\end{document}
